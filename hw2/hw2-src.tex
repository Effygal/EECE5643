% !TEX TS-program = latex
\documentclass[11pt]{article}

% set this flag to 0 to remove comments
\def\comments{1}

\usepackage{amsmath,amssymb,amsthm,listings}
\usepackage{bbm}
\usepackage{paralist}
\usepackage[linesnumbered,ruled,vlined]{algorithm2e}
\usepackage{authblk}
	\renewcommand{\Authsep}{\qquad}
	\renewcommand{\Authand}{\qquad}
	\renewcommand{\Authands}{\qquad}
	\renewcommand\Affilfont{\itshape\small}
\usepackage[left=1.25in,right=1.25in,top=1.25in,bottom=1.25in]{geometry}
\usepackage[bookmarks=false]{hyperref}
    \hypersetup{
        linktocpage=true,
        colorlinks=true,				
        linkcolor=DarkBlue,				
        citecolor=DarkBlue,				
        urlcolor=DarkBlue,			
    }
\usepackage[tt=false]{libertine}
    \usepackage[libertine]{newtxmath}
    \usepackage[T1]{fontenc}
    \renewcommand{\baselinestretch}{1.00}
\usepackage{lipsum}
\usepackage{microtype}
\usepackage{multirow}
\usepackage{enumitem}
\usepackage{nicefrac}
\usepackage{tikz}
	\usetikzlibrary{positioning}
	\definecolor{DarkGreen}{rgb}{0.2,0.6,0.2}
	\definecolor{DarkRed}{rgb}{0.6,0.2,0.2}
	\definecolor{DarkBlue}{rgb}{0.2,0.2,0.6}
	\definecolor{DarkPurple}{rgb}{0.4,0.2,0.4}   
\usepackage{url}
\usepackage{verbatim}
\usepackage{wrapfig}
\usepackage{framed}
\usepackage{graphicx}
% \usepackage{ulem}
% \normalem

   
% comments
\setlength\marginparwidth{62pt}
\setlength\marginparsep{5pt}
\ifnum\comments=1
    \newcommand{\mynote}[2]{{\marginpar{\color{#1}\sf \tiny #2}}}
    \newcommand{\mynoteinline}[2]{{\color{#1} \sf \small #2}}
\else
\newcommand{\mynote}[2]{}
    \newcommand{\mynoteinline}[2]{}
\fi
\newcommand{\jnote}[1]{\mynote{blue}{JU: #1}}
\newcommand{\as}[1]{\mynote{red}{ADS: #1}}
\newcommand{\anote}{\as}
\newcommand{\asinline}[1]{\mynoteinline{red}{ADS: #1}}

\renewcommand{\epsilon}{\eps}

% fixing left-right spacing
\let\originalleft\left
\let\originalright\right
\renewcommand{\left}{\mathopen{}\mathclose\bgroup\originalleft}
\renewcommand{\right}{\aftergroup\egroup\originalright}

% math macros
\newcommand{\ex}[2]{{\ifx&#1& \mathbb{E} \else \underset{#1}{\mathbb{E}} \fi \left(#2\right)}}
\newcommand{\pr}[2]{{\operatorname*{\mathbb{P}}_{#1} \paren{#2}}}
\newcommand{\var}[2]{{\ifx&#1& \mathrm{Var} \else \underset{#1}{\mathrm{Var}} \fi \left(#2\right)}}

\newcommand{\mypar}[1]{\medskip\textbf{#1}}

\newcommand{\from}{:}
\newcommand{\poly}{\mathrm{poly}}
\newcommand{\polylog}{\mathrm{polylog}}
\newcommand{\eps}{\varepsilon}

\newcommand{\ind}{\mathbb{I}}
\newcommand{\pmo}{\{\pm 1\}}
\newcommand{\zo}{\{0,1\}}
\newcommand{\bit}[1]{{\zo^{#1}}}
\newcommand{\cond}[1]{\vert_{#1}}
\newcommand{\norm}[2]{\|#1\|_{#2}}
\newcommand{\abs}[1]{{\left | {#1} \right|}}
\newcommand{\half}{\frac{1}{2}}
\newcommand{\nicehalf}{\nicefrac{1}{2}}
\newcommand{\set}[1]{\left\{ #1 \right\}}
\newcommand{\cardset}[1]{\left| \set{#1} \right|}
\newcommand{\paren}[1]{{\left ( {#1} \right)}}
\newcommand{\bparen}[1]{{\big( {#1} \big)}}
\newcommand{\Bparen}[1]{{\Big( {#1} \Big)}}
\newcommand{\bracket}[1]{{\left [ {#1} \right]}}
\newcommand{\ip}[1]{{\left \langle {#1} \right\rangle }}

\newcommand{\distance}{\mathrm{d}}
\newcommand{\dtv}{\distance_{\mathrm{TV}}}
\newcommand{\dkl}{\distance_{\mathrm{KL}}}
\newcommand{\dhe}{\distance_{\mathrm{H^2}}}
\newcommand{\dcs}{\distance_{\mathrm{\chi^2}}}

\DeclareMathOperator*{\argmin}{arg\,min}
\DeclareMathOperator*{\argmax}{arg\,max}

\newcommand{\N}{\mathbb{N}}
\newcommand{\R}{\mathbb{R}}
\newcommand{\Z}{\mathbb{Z}}

\newcommand{\cA}{\mathcal{A}}
\newcommand{\cB}{\mathcal{B}}
\newcommand{\cC}{\mathcal{C}}
\newcommand{\cD}{\mathcal{D}}
\newcommand{\cE}{\mathcal{E}}
\newcommand{\cF}{\mathcal{F}}
\newcommand{\cG}{\mathcal{G}}
\newcommand{\cH}{\mathcal{H}}
\newcommand{\cI}{\mathcal{I}}
\newcommand{\cJ}{\mathcal{J}}
\newcommand{\cK}{\mathcal{K}}
\newcommand{\cL}{\mathcal{L}}
\newcommand{\cM}{\mathcal{M}}
\newcommand{\cN}{\mathcal{N}}
\newcommand{\cO}{\mathcal{O}}
\newcommand{\cP}{\mathcal{P}}
\newcommand{\cQ}{\mathcal{Q}}
\newcommand{\cR}{\mathcal{R}}
\newcommand{\cS}{\mathcal{S}}
\newcommand{\cT}{\mathcal{T}}
\newcommand{\cU}{\mathcal{U}}
\newcommand{\cV}{\mathcal{V}}
\newcommand{\cW}{\mathcal{W}}
\newcommand{\cX}{\mathcal{X}}
\newcommand{\cY}{\mathcal{Y}}
\newcommand{\cZ}{\mathcal{Z}}


\newcommand{\bc}{\mathbf{c}}
\newcommand{\bp}{\mathbf{p}}
\newcommand{\bw}{\mathbf{w}}
\newcommand{\by}{\mathbf{y}}

\newcommand{\defeq}{\stackrel{{\mbox{\tiny def}}}{=}}

\newtheorem{thm}{Theorem}
    \newtheorem{clm}[thm]{Claim}
    \newtheorem{lem}[thm]{Lemma}
    \newtheorem{prop}[thm]{Proposition}
    \newtheorem{cor}[thm]{Corollary}
    \newtheorem{fact}[thm]{Fact}
    \newtheorem{claim}[thm]{Claim}
\theoremstyle{definition}
    \newtheorem{defn}[thm]{Definition}
    \newtheorem{example}[thm]{Example}
    \newtheorem{rem}[thm]{Remark}
    \newtheorem{exer}[thm]{Exercise}

\newcommand{\HWtitle}[2]{\begin{figure}[t!]{\bfseries \Large \color{DarkBlue}  \noindent EECE 5643 \hfill Spring 2023} \\[0.2em] {\bfseries \Large \color{DarkBlue} Homework #1: Due {#2}} \\[1em] \\[1ex] \end{figure}}

\begin{document}
\renewcommand{\labelenumii}{{\bfseries \em \arabic{enumi}.\arabic{enumii}}}
\newcommand{\problemitem}{\renewcommand{\labelenumi}{{\bfseries \em Problem \arabic{enumi}}}\item}
\newcommand{\solutionitem}{\renewcommand{\labelenumi}{{\bfseries \em Solution \arabic{enumi}}}\addtocounter{enumi}{-1}\item}
\newenvironment{solution}{\par\color{DarkBlue}}{\par}
{\noindent \textbf{Yirong Wang} }

\paragraph{Homework 2}
\begin{enumerate}[leftmargin=0pt, itemsep=3ex]

    \problemitem Ex1.2.2
        \begin{enumerate}
        \item (a) for 1000 jobs:
   \\ average interarrival time = 9.87
   \\ average service time = 7.12
   \\ average delay = 18.59
   \\ average wait = 25.72
   \\ maximum delay = 118.76
   \\ num of jobs in the service node at (t=400) = 7
   \\ proportion of jobs delayed = 0.72
    
    \item (b) maximum delay = 118.76
    \item (c) num of jobs in the service node at (t=400) = 7.
    \\To intergrate the indicator function $\varphi_i(t)$ over the job domain is the process that counts the number of jobs  that arrives before $t$ and leaves after $t$ for $\forall i \in \{1...n\}$.
    $$l(t) = \sum_{i=1}^{n} \varphi_i(t)$$ Let $t=400$, the simulation results indicates that $$l(t=400) = 7$$ 
    \item (d) The simulation results shows that the proportion of jobs delayed = 0.72. Given that $$\overline{x} = \frac{1}{c_n} \int_{0}^{c_n}x(t)dt$$
    and, $$\int_{0}^{c_n}x(t)dt = \sum_{i=1}^{n}s_i$$
    therefore, $$\overline{x} = \frac{n \overline{s}}{c_n} = \frac{1000 \times 7.12}{9897.22} \approx 0.72$$
    Observe that this number is equal to the proportion of job delayed. The explanation is that ...
    
    \end{enumerate}

    \problemitem Ex.1.2.6
    
    \begin{enumerate}
        \item (a)
        for 500 jobs
        \\average service time = 3.03
        \\server's utilization = 0.74
        \\traffic intensity = 0.74
        \item (b)   
        $$
        s_i = c_i - a_i - d_i
        $$
        where
        $$
        d_i = c_{i-1} - a_i
        $$
        therefore,
        $$
        s_i = c_i - a_i - c_{i-1} + a_i = c_i - c_{i-1}
        $$
    \end{enumerate}

    \problemitem Ex.2.3.4
        \begin{enumerate}
        \item (a)
        Let $X \in \mathbb{Z}$ denote the r.v. of the sum of the two up-faces,
        based on 1000000 replications the estimated probabilities are:
        \begin{align*}
        \\P[X=2] = 0.006
        \\P[X=3] = 0.024
        \\P[X=4] = 0.048
        \\P[X=5] = 0.071
        \\P[X=6] = 0.095
        \\P[X=7] = 0.142
        \\P[X=8] = 0.165
        \\P[X=9] = 0.142
        \\P[X=10] = 0.119
        \\P[X=11] = 0.095
        \\P[X=12] = 0.094
        \end{align*}
        \item (b)
        \begin{align*}
            P[X=7] &= P[1 \land 6]+P[2 \land 5]+P[3 \land 4]
             \\ &= \frac{1}{13}\times \frac{4}{13} \times 2 + \frac{2}{13}\times \frac{2}{13} \times 2 + \frac{2}{13}\times \frac{2}{13} \times 2 
             \\ &= \frac{8}{169} + \frac{8}{169} + \frac{8}{169}
             \\ &= \frac{24}{169}
             \\ & \approx 0.142
        \end{align*}
        \end{enumerate}

    \problemitem Ex.2.3.5
    
        \begin{enumerate}
            \item (a) Let $\overrightarrow{v}$ and $\overrightarrow{w}$ denote two randomly selected 2D-vectors land on the circumference of the circle centering at the origin of a two-dimensional coordinate plane, of radius $\rho = 1$.
            Let $d(\overrightarrow{v},\overrightarrow{w})$ denote the distance between $\overrightarrow{v}$ and $\overrightarrow{w}$.
    Based on 100000 replications of Monte Carlo simulation, the estimated probabilities are:
    $$P[d(\overrightarrow{v},\overrightarrow{w}) \geqslant \rho] = 0.667$$
    \item (b) The event that $d(\overrightarrow{v},\overrightarrow{w})$ is independent with the value of $\rho$.
        \end{enumerate}
    
\end{enumerate}
\end{document}
